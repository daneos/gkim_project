\documentclass[12pt,a4paper]{article}
\usepackage[utf8]{inputenc}
\usepackage{polski}
\usepackage{graphicx}

\newcommand{\HRule}{\rule{\linewidth}{0.5mm}}
\newcommand{\code}[1]{\texttt{#1}}
\newcommand{\note}{\textbf{Uwaga:} }

\begin{document}
	\begin{titlepage}
\begin{center}

\includegraphics{./pk.jpg}~\\[1cm]

\textsc{\LARGE Politechnika Krakowska}\\[1.5cm]

\textsc{\Large Grafika Komputerowa i Multimedia}\\[0.5cm]

\HRule \\[0.4cm]
{ \huge \bfseries Format HCI \\[0.4cm] }
\HRule \\[1.5cm]

\noindent
\begin{minipage}[t]{0.4\textwidth}
\begin{flushleft}
\emph{Autorzy:}\\
Grzegorz \textsc{Kowalski}\\
Bartosz \textsc{Zielnik}\\
Piotr \textsc{Mańkowski}\\
Dariusz \textsc{Szyszlak}
\end{flushleft}
\end{minipage}
\begin{minipage}[t]{0.4\textwidth}
\begin{flushright}
\emph{Prowadzący:}\\
dr inż. Damian \textsc{Grela}\\
mgr Kamil \textsc{Nowakowski}
\end{flushright}
\end{minipage}

\vfill

{\large \today}\\
{\tiny Stworzono za pomocą \LaTeXe}

\end{center}
\end{titlepage}

	\section{Obsługa programu}
		\subsection{Kompilacja}
			\subsubsection{GNU/Linux}
				Aby skompilować program w systemie Linux należy mieć zainstalowany pakiet \code{libsdl2-dev} (lub odpowiednik w używanej dystrybucji).
				Kompilacja przeprowadzana jest automatycznie przez skrypt \code{make.sh} dołączony do źródeł projektu.
			\subsubsection{Windows}
				Aby skompilować program w systemie Windows należy sformatować dysk twardy, zainstalować Linuxa oraz wykonać kroki z punktu 1.1.1.
		\subsection{Obsługa}
			Program obsługiwany jest z linii poleceń. Przyjmuje różną ilość argumentów, w zależności od akcji, którą chcemy wykonać.\\
			Wywołanie programu z trzema argumentami powduje uruchomienie konwersji z \code{.bmp} na \code{.hci}.\\
			Jako pierwszy argument należy podać typ kompresji, jako drugi wejściowy plik \code{.bmp}, a jako trzeci wyjściowy plik \code{.hci}.\\
			Program obsługuje 3 rodzaje kompresji:
			\begin{itemize}
				\item \code{p} -- \textbf{byte-packing} -- algorytm konwertujący każdy kanał na 4 bity i pakujący 2 kanały na bajt
				\item \code{h} -- \textbf{Huffman} -- algorytm kodowania Huffmana
				\item \code{l} -- \textbf{LZ77} -- algorytm kompresji Lempel-Ziv 77
			\end{itemize}
			\note{
				Algorytm Huffmana pracuje na niespakowanych danych 4-bitowych.\\
				Algorytm LZ77 pracuje na spakowanych danych 4-bitowych (na wyniku algorytmu byte-packing).
			}\\
			Do typu kompresji można dodać opcję \code{g}, która powoduje przekonwertowanie obrazu do skali szarości.\\
			Przykładowe wywołanie programu:\\
			\code{\$ hciconv h obrazek.bmp obrazek.hci}\\
			spowoduje umieszczenie w pliku \code{obrazek.hci} grafiki z pliku \code{obrazek.bmp} skompresowanej kodem Huffmana.\\
			\code{\$ hciconv lg obrazek.bmp obrazek.hci}\\
			spowoduje skompresowanie obrazka algorytmem LZ77 i jego konwersję do skali szarości.\\
			Uruchomienie programu z dwoma argumentami powoduje dekompresję pliku \code{.hci} do pliku \code{.bmp}.
			Nie trzeba podawać algorytmu jakim plik został zakodowany, ponieważ program potrafi sam wykryć rodzaj kompresji.\\
			Przykładowe wywołanie programu:\\
			\code{\$ hciconv obrazek.hci obrazek.bmp}\\
			spowoduje zdekodowanie pliku \code{obrazek.hci} i zapisanie wyniku w pliku \code{obrazek.bmp}.

	\section{Opis formatu}
		Format HCI (Hopefully Compressed Image) to graficzny format binarny.\\
		Plik w tym formacie składa się z kilku elementów: 12-bajtowego nagłówka oraz danych kodowania.
		\subsection{Nagłówek}
			Nagłówek pliku HCI składa się z następujących elementów:
			\begin{itemize}
				\item identyfikator (3 bajty, zawsze równy ciągowi znaków "HCI") -- pozwala stwierdzić czy dany plik jest plikiem w formacie HCI
				\item szerokość obrazka (4 bajty)
				\item wysokość obrazka (4 bajty)
				\item typ kompresji (1 bajt) -- pozwala rozpoznać algorytm, którym zakodowane są dane
			\end{itemize}
		\subsection{Dane kodowania}
			Format danych zależy od użytego algorytmu, ale ogólnie dane te składają się z nagłówka oraz właściwych, zakodowanych danych.
			\subsubsection{Byte-packing}
				W nagłówku algorytmu byte-packing znajduje się jedynie 4-bajtowa długość danych.
			\subsubsection{Huffman}
				Nagłówek algorytmu Huffmana składa się z:
				\begin{itemize}
					\item 2-bajtowego rozmiaru słownika
					\item 2-bajtowej długości najdłuższego kodu Huffmana
					\item słownika do dekodowania
					\item 4-bajtowej wielkości zakodowanego obrazu potrzebnej do ustawienia rozmiaru bufora dekodera
				\end{itemize}
			\subsubsection{LZ77}
				5-bajtowy nagłówek algorytmu LZ77 zawiera dwa pola: 1-bajtowy rozmiar słownika oraz 4-bajtowy rozmiar rozkompresowanych danych, używany do
				poprawnego ustawienia buforów dekompresora.

\end{document}
